
\section{Running \ISETL{}}

\ISETL{} is an interpreted, interactive version of the programming language
\SETL\@.
It is invoked by typing a command line with
the executable name, say \[isetl\], along with optional file
names that are discussed below.\footnote{The \XX{Macintosh} version
is clickable.  
}

There is no compiler for \ISETL{}\@.
When \ISETL{} is running,
it prompts for input with the character ``\[>\]''.
Input consists of a sequence of expressions
(each terminated by a semicolon ``\[;\]''), statements, and programs.
Each input is acted upon as soon as it is entered.
These actions are explained below.
In the case of expressions,
the result includes its value being printed.
If you have not completed your entry,
you will receive the prompt ``\[>>\]'',\index{prompts}
indicating that more is expected.

\begin{enumerate}

\item
\index{exit}\index{quit}\key{!quit}
\ISETL{} is exited by typing ``\[!quit\]''.
It may also be exited by ending the standard input.
In \XX{Unix}, this is done by typing ctrl-D\@.
In MS-DOS, ctrl-Z and ctrl-D will work.

\item
A common mistake is omitting the semicolon after an expression.
\ISETL{} will wait until it gets a semicolon before proceeding.
The doubled prompt ``\[>>\]'' indicates that \ISETL{} is expecting more
input.

\item
\ISETL{} can get its input from sources other than the standard input.

\begin{enumerate}
    \item
    \index{.isetlrc tt} \index{isetl.ini tt}\label{.ini}
    If there is an initialization file\footnote{
        Initialization files are called either \[.isetlrc\] or \[isetl.ini\].
	The file is looked for in:
	\begin{enumerate}
	\item the current directory
	\item the directory containing \[isetl.exe\] (MSDOS and Mac)
	\item the home directory (Unix, VMS) or root (MSDOS)
	\item in the symbol \[ISETLINI:\] (VMS only)
	\end{enumerate}
	Only one initialization file is read.
	The same pattern is searched for the \[ile\] initialization file.
	}
    in the current directory,
    then the first thing \ISETL{} will do is read this file.

    \item
    Next, if the \XX{command line} has any file names listed, \ISETL{} will
    read each of these in turn.\footnote{
	This feature is system dependent.
	To provide this feature in \XX{VMS}, you must define
	\[isetl~:==~\$your\$disk:[your.dir]isetl.exe\] in your
	\[login.com\].  The leading \[\$\] makes
	this a {\em foreign command\/}.  The rest is the
	complete path to the executable version of \ISETL{}\@.}

    Thus, if the command line reads,

    \DS isetl file.1 blue green \DE


    \ISETL{} will first read from ``\[.isetlrc\]'' if it exists,
    and then from ``\[file.1\]'', then ``\[blue\]'',
    and then ``\[green\]''.
    Finally, it is ready for input from the terminal.

    \item
    If there is a file available --- say ``\[file.2\]'' --- and
    \ISETL{} is given (at any time), the following line of input,

    \key{!include}
    \DS !include file.2 \DE

    then it will take its input from ``\[file.2\]''
    before being ready for any further input.
    The material in such a file is treated
    {\em exactly as if it were typed directly at the keyboard,\/}
    and it can be followed on subsequent lines
    by any additional information that
    the user would like to enter.

    Consider the following (rather contrived) example:
    Suppose that the file ``\[file.3\]'' contained the following data:

    \DS 5, 6, 7, 3, -4, "the" \DE

    Then if the user typed,

\DS
>      seta := \{\\
>>     !include file.3\\
!include file.3 completed\\
>>     , x \};\\
\DE

    the effect would be exactly the same as if the user had entered,

\DS
>      seta := \{5, 6, 7, 3, -4, "the", x\};\\
\DE

    The line ``\[!include file.3 completed\]''
    comes from \ISETL{} and is always printed after an ``\[!include\]''.
\end{enumerate}

\item
Comments\index{comments (zdollar tt)}\index{dollar sign}

If a dollar sign ``\[\$\]'' appears on a line,
then everything that appears until
the end of the line is ignored by \ISETL{}\@.

\item
After a program or statement has executed,
the values of global variables persist.
The user can then evaluate expressions in terms of these variables.
(See section~\ref{funcs} for more detail on scope.)
\end{enumerate}


\section{Characters, Keywords, and Identifiers}\label{char-key}

\subsection{Character Set}  

\index{character set}
The following is a list of characters used by \ISETL{}\@.

\CS\tt
\verb|@ |\verb@[ ] ; : = | { } ( ) . # ? * / + - _ " < > % ~ ,@\\
\mbox{}\\
a {\rm ---} z         A {\rm ---} Z        0 {\rm ---} 9\\
\CE

\noindent
In addition, the following character-pairs are used.

\CS
\verb|:=    ..    **    /=    <=    >=    ->|
\CE

The characters ``\[:\]'' and ``\[|\]'' may be used interchangably.

\subsection{Keywords}

The following is a list of \ISETL{} \XX{keywords}.
\key{and}	\key{newat}	\key{subset}
\key{div}	\key{not}	\key{take}      \key{frome}
\key{do}	\key{func}	\key{notin}	\key{then}
\key{else}	\key{iff}	\key{of}	\key{to}
\key{elseif}	\key{if}	\key{om}	\key{true}
\key{end}	\key{impl}	\key{opt}	\key{union}
\key{exists}	\key{inter}	\key{or}	\key{value}
\key{false}	\key{in}	\key{print}	\key{where}
\key{for}	\key{less}	\key{program}	\key{while}
\key{forall}	\key{local}	\key{read}	\key{with}
\key{mod}	\key{return}	\key{readf}	\key{printf}
\key{write}	\key{writeln}	\key{from}	\key{fromb}
\DS
\begin{tabular}{ l l l l l l l}
and&	false&	iff&	not&	program&true\\
div&	for&	impl&	notin&	read&	union\\
do&	forall&	in&	of&	readf&	value\\
else&	from&	inter&	om (OM)&return&	where\\
elseif&	fromb&	less&	opt&	subset&	while\\
end&	frome&	local&	or&	take&	with\\
exists&	func&	mod&	print&	then&	write\\
&	if&	newat&	printf&	to&	writeln
\end{tabular}
\DE

\subsection{Identifiers}

    \begin{enumerate}

    \item
    An \XX{identifier} is a sequence of alphanumeric characters along with the
    underscore, caret, and prime --- ``\[\_\caret '\]''.
    It must begin with a letter.
    Upper or lower case may
    be used, and \ISETL{} preserves the distinction.
    (I.e.: \[a\_\,good\_\,thing\] and \[A\_\,Good\_\,Thing\]
    are both legal and are different.)

    \item
    An identifier serves as a variable and can take on a value of any \ISETL{}
    data type.
    The type of a variable is entirely determined by the value
    that is assigned to it and changes when a value of a different type is
    assigned.
    \end{enumerate}

\section{Simple Data Types}

    \subsection{Integers}
    \begin{enumerate}
    \item
    \index{integer} \index{number}
    There is no limit to the size of integers.\footnote{
	No {\em practical\/} limit.  
	Actually limited to about 20,000 digits per integer.}
    \item
    An integer constant is a sequence of one or more digits.
    It represents an unsigned integer.
    \item
    On input and output, long integers may be broken to accommodate 
    limited line length.  
    A backslash (``\[$\backslash$\]'') at the end of a sequence of
    digits indicates that the integer is continued on the next line.

\begin{indented}\begin{verbatim}
>       123456\
>>        789;
123456789;
\end{verbatim}\end{indented}


    \end{enumerate}

    \subsection{Rationals}
    \begin{enumerate}
    \item Rationals are only created when the directive \[!rational~on\]
          has been used.
    \item Rationals are created by dividing integers.
    \item Arithmetic remains rational as long as possible.
    \end{enumerate}

    \subsection{Floating\_\,Point Numbers}
    \begin{enumerate}
    \item
    \index{floating-point number} \index{number}
    The possible range of floating\_\,point numbers is machine dependent.
    At a minimum, the values will have 5 place accuracy, with a range of
    approximately $10^{38}$.
    \item
    A floating\_\,point constant is a sequence of zero or more digits,
    followed by a decimal point,
    followed by zero or more digits.
    Thus, \[2.0\], \[.5\], and \[2.\] are all legal.
    
    A floating\_\,point constant may be followed by an exponent.
    An exponent consists of one of the characters ``\[e\]'',
    ``\[E\]'', ``\[f\]'', ``\[F\]'' followed by a signed or unsigned
    integer.
    The value of a floating\_\,point constant
    is determined as in scientific notation.
    Hence, for example, \[0.2\], \[2.0e-1\], \[20.0e-2\] are all equivalent.
    As with integers, it is unsigned.

    \item
    Different systems use different printed representations when
    floating point values are out of the machine's range.
    For example, when the value is too large, the \XX{Macintosh}
    prints ``+++++'' and the Sun prints ``Infinity''.
    \end{enumerate}

    \subsection{Booleans}
    \begin{enumerate}
    \item
    A \XX{Boolean} constant is one of the keywords 
    \[true\] or \[false\],\key{true}\key{false}
    with the obvious meaning for its value.
    \end{enumerate}

    \subsection{Strings}
    \begin{enumerate}
    \item
    A \XX{string} constant is any sequence of
    characters preceded and followed by a double quote, ``\["\]''.
    A string may not be split across lines.
    Large strings may be constructed using the
    operation of concatenation.
    Strings may also be surrounded by single quotes, ``\['\]''.

The backslash convention may be used to enter special characters.
When pretty-printing, these conventions are used for output.
In the case of formated output, the special characters are printed.

\CS
\begin{tabular}{l l}
$\backslash$b&	backspace\\
$\backslash$f&	formfeed (new page)\\
$\backslash$n&	newline (prints as CR-LF)\\
$\backslash$q&	double quote\\
$\backslash$r&	carriage return (CR)\\
$\backslash$t&	tab\\
$\backslash${\it octal}&	character represented by {\it octal}\\
\mbox{}&			Refer to an ASCII chart for meaning.\\
$\backslash${\it other}&	{\it other} --- may be any character\\
\mbox{}&			not listed above.
\end{tabular}
\CE

In particular, \["$\backslash\backslash$"\] is a single backslash.
You may type, \["$\backslash$""\] for double quote, but the pretty
printer will print as \["$\backslash$q"\].
ASCII values are limited to \['$\backslash$001'\] to
\['$\backslash$377'\].

\begin{verbatim}
>       %+ [char(i): i in [1..127]];
"\001\002\003\004\005\006\007\b\t\n\013\f"
+"\r\016\017\020\021\022\023\024\025\026"
+"\027\030\031\032\033\034\035\036\037 !"
+"\q#$%&'()*+,-./0123456789:;<=>?@ABCDEF"
+"GHIJKLMNOPQRSTUVWXYZ[\\]^_`abcdefghijk"
+"lmnopqrstuvwxyz{|}~\177";
\end{verbatim}


    \end{enumerate}

    \subsection{Atoms}
    \begin{enumerate}
    \item
    \index{atom tt}
    Atoms are ``abstract points''.
    They have no identifying properties other than their individual existence.

    \item
    The keyword \TT{newat} has as its value an atom never before
    seen in this session of \ISETL{}\@.
    \end{enumerate}
    
    \subsection{Files}
    \begin{enumerate}
    \item\index{file}
    A file is an \ISETL{} value that
    corresponds to an external file in the operating system environment.

    \item
    They are created as a result of applying one of the pre-defined
    functions \[openr\], \[opena\], \[openw\].
    (See section~\ref{file-func}.)
    \end{enumerate}

    \subsection{Undefined}
    \begin{enumerate}
    \item
    The data type \XX{undefined}
    has a single value --- \TT{OM}\@.
    It may also be entered as \[om\].

    \item
    Any identifier that has not been
    assigned a value has the value \[OM\]\@.
    \end{enumerate}



\section{Compound Data Types}

\subsection{Sets}
    \begin{enumerate}
    \item
    Only finite sets may be represented in \ISETL{}\@.
    The elements may be of any type, mixed heterogeneously.
    Elements occur at most once per \XX{set}.

    \item
    \TT{OM} may not be an element of a set.
    Any set that would contain \[OM\] is considered to be undefined.

    \item
    The order of elements is not significant
    in a set and printing the value
    of a set twice in succession could display the elements in
    different orders.

    \item
    Zero or more expressions, separated by commas and enclosed in
    braces (``\[\{\]'' and ``\[\}\]'') evaluates to
    the set whose elements are the values of the
    enclosed expressions.

    \index{empty\\set ({} tt,zempty)}
    Note that as a special case, the empty set is denoted by \[\{~\}\].

    \item
    There are syntactic forms, explained in the grammar, for a finite set
    that is an \XX{arithmetic progression} of integers,
    and also for a finite set
    \index{set former}
    obtained from a set former in standard mathematical notation.

    For example, the value of the following expression

\CS
\verb@{ x+y : x,y in {-1,-3..-100} | x /= y };@
\CE

    is the set of all sums of two different odd negative integers
    larger than $-100$.
    \end{enumerate}

\subsection{Tuples}

    \begin{enumerate}
    \item
    A \XX{tuple} is an infinite sequence of components,
    of which only a finite number are defined.
    The components may be of any type, mixed heterogeneously. The values of
    components may be repeated.

    \item
    \[OM\] is a legal value for a component.

    \item
    The order of the components of a tuple is significant.
    By treating the tuple as a function over the positive integers,
    you can extract individual components and 
    contiguous subsequences (slices) of the tuple.

    \item
    Zero or more expressions, separated by commas and enclosed in
    square brackets (``\[[\]'' and ``\[]\]'') evaluates to
    the tuple whose defined components are the values of the
    enclosed expressions.

    \index{empty\\tuple ([] tt)}
    Note that as a special case, the empty tuple is denoted by \[[~]\].
    This tuple is undefined everywhere.

    \item
    The syntactic forms for tuples of finite arithmetic progressions
    and tuple formers are similar to those provided for sets.
    The only difference is the use of square,
    rather than curly, brackets.

    \item
    The length of a tuple is the largest index (counting from 1) for which
    a component is defined (that is, is not equal to \[OM\]).
    It can change at run-time.

    \item
    Tuples usually are indexed starting at 1,
    but they can have different starting indices.
    See page~\pageref{tuple-at} and page~\pageref{tuple-func} for definitions.

    \item
    Tuples created by a \[FORMER\] have the default origin.  See
    \[origin\] for how to redefine the default.

    \item
    Tuples that result from operations on other tuples inherit their origin.
    Generally, the result inherits the origin of the leftmost tuple argument.
    \end{enumerate}

\subsection{Maps}

Maps form a subclass of sets.

    \begin{enumerate}

    \item \index{map}
    A \[map\] is a set that is either empty
    or whose elements are all ordered pairs.
    An ordered pair is a tuple
    whose first two components and no others are defined.

    \item
    There are two special operators for evaluating a map at a point in its
    domain.
    Suppose that \[F\] is a map.

	\begin{enumerate}

	\item
	\[F(EXPR)\] will evaluate to the value of the second component
	of the ordered pair whose first component
	is the value of \[EXPR\], provided there is exactly
	one such ordered pair in \[F\];
	if there is no such pair, it evaluates to \[OM\];
	if there are many such pairs, an error is reported.

	\item
	\[F\{EXPR\}\] will evaluate to the set of all values
	of second components of ordered pairs in \[F\]
	whose first component is the value of \[EXPR\]\@.
	If there is no such pair, its value is the empty set.
	\end{enumerate}


    \item
    A map in which no value appears more 
    than once as the first component of an
    ordered pair is called a {\em single-valued map\/} or 
    \TT{smap};
    otherwise, the map is called a {\em multi-valued map\/} or 
    \TT{mmap}.
    \end{enumerate}


\section{Funcs}\label{funcs}

    \index{func tt sff}
    \index{function sff}
    \index{proc tt sff}
    \index{procedure sff}
    \begin{enumerate}
    \item
    A \TT{func} (\TT{proc}) is an \ISETL{} value that may be applied 
    to zero or more values passed to it as arguments.
    It then returns a value specified by the
    definition of the func. (\[procs\] return an non-printing version
    of OM and should only be used in the place of statements.)
    Because it is a value, an \ISETL{} func can
    be assigned to an identifier, passed as an argument, etc.
    Evaluation of an \ISETL{} func can have side-effects determined by the
    statements in the definition of the func.
    Thus, it also serves the purpose of what is often called a 
    \XX{procedure}.

    \item
    \index{return tt}%
    The \[return\] statement is only meaningful inside a func.
    Its effect is to terminate execution of
    the func and return a value to the caller.
    The form ``\[return {\it expr};\]'' returns the value of {\it expr};
    ``\[return;\]'' returns \[OM\].

    \ISETL{} inserts \[return;\] just before the \[end\] of every func.


    \item
    A func is the computational 
    representation of a \XX{function}, as a map
    is the ordered pair representation, and a tuple is the sequence
    representation.
    Just as tuples and maps may be modified at a point 
    by assignment, so can funcs.
    However, if the value at a point is structured, you may not modify
    that at a point as well.

\begin{indented}
\begin{verbatim}
>       x := func(i);
>>               return char(i);
>>           end;
>       x(97);
"a";
>       x(97) := "q"; 
>       x(97);
"q";
>       x(97)(1) := "abc";
! Error: Only one level of selection allowed
\end{verbatim}
\end{indented}

     \[x\] may be modified at a point.
     The assignment to \[x(97)\] is legal.
     However, the following assignment is not supported at this time,
     because you are trying to modify the structure of the value
     returned.

    \item
    A number of functions have been \XX{pre-defined}
    as funcs in \ISETL{}\@.
    A list of their definitions is given in section~\ref{predef}.
    These are not keywords and may be changed by the user.
    They may not be modified at a point, however.

    \item
    It is possible for the user to define her/his own func.
    This is done with the following syntax:

\begin{indented}
\begin{verbatim}
func(list-of-parameters);
    local list-of-local-ids;
    value list-of-global-ids;
    statements;
end
\end{verbatim}
\end{indented}

Alternately, one may write
\begin{indented}
\begin{verbatim}
: list-of-parameters -> result :
\end{verbatim}
\end{indented}
if the function simply consists of evaluating an expression.

	\begin{enumerate}

	\item
	The declaration of \TT{local} ids
	may be omitted if no local variables are needed.
	The declaration of \TT{value} ids 
	represents global variables
	whose {\em current values are to be remembered\/}
	and used at the time of function invocation;
	these may be omitted if not needed.
	The list-of-parameters may be empty,
	but the pair of parentheses must be present.

\pagebreak[1]
	\item
	 Parameters and local-ids are local to the func.
	 See below for a discussion of scope.

\pagebreak[1]
	\item
	The syntax described above is for an {\em expression\/}
	of type func.
	As with any expression,
	it may be evaluated, but the value has no name.
	Thus, the definition will typically be part of an assignment
	statement or passed as a parameter.
	As a very simple example, consider:

\begin{indented}
\begin{verbatim}
cube_plus := func(x,y);
                 return x**3 + y;
             end;
\end{verbatim}
\end{indented}

	After having executed this input, \ISETL{} will evaluate an
	expression such as \[cube\_\,plus(2,5)\] as \(13\).

\pagebreak[1]
	\item
	\index{parameter}\index{call by value}\index{optional parameters}
	Parameters are passed by value.
	It is an error to pass too many or too few arguments.
	It is possible to make some parameters {\em optional\/}.

	\CS
	\verb|f := func(a,b,c opt x,y,z); ... end;|
	\CE

	\[f\] can be called with 3, 4, 5, or 6 arguments.
	If there are fewer than 6 arguments, the missing arguments are
	considered to be \[OM\].

\pagebreak[1]
	\item
	\index{scope}
	Scope is \XX{lexical} (\XX{static}) with \XX{retention}.
	{\em Lexical\/} means that
	references to \XX{global variables} are determined by
	where the func was created,
	not by where it will be evaluated.
	{\em Retention\/} means that even
	if the scope that created the func has been exited,
	its variables persist and can be used by the func.

	By default, references to global variables will use the value
	of the variable at the time the function is invoked.
	\index{value declaration}
	The \[value\] declaration causes the value of the
	global variable {\em at the time the func is created\/} to
	be used.

\pagebreak[1]
	\item
	Here is a more complicated example of the use of func.
	As defined below, \[compose\] takes two functions as arguments
	and creates their functional composition.
	The functions can be any \ISETL{}
	values that may be applied to a single
	argument; e.g. func, tuple, smap.

\begin{indented}
\begin{verbatim}
compose := func(f,g); 
               return :x -> f(g(x)) :
           end;
twice :=   :a -> 2*a: ;
times4 :=  compose(twice,twice);
\end{verbatim}
\end{indented}
	
	Then the value of \[times4(3)\] would be 12.  	The value of
	\[times4\] needs to refer to the values of \[f\] and \[g\],
	and they remain accessible to \[times4\], 	even though
	\[compose\] has returned.

\pagebreak[1]
	\item
	Finally, here is an example of\index{function\\modified at a point}
	functions modified at a point and
	functions that capture the current value of a global.

\begin{indented}
\begin{verbatim}
f := func(x);
         return x + 4;
     end func;
gs := [ func(x); value N; return x+3*N; end
        : N in [1..3] ];
f(3) := 21;
\end{verbatim}
\end{indented}

	After this is executed, \[f(1)\] is 5, \[f(2)\] is 6,
	but \[f(3)\] is 21.
	\[gs(2)(4)\] is 10 (\[4+3*2\]).
%	The function defined below by func and assigned to
%	the identifier \[curry\] takes as its single
%	parameter any value \[F\] for which \[F(x,y)\] makes sense
%	(e.g., a func, a map, etc.),
%	and returns a func \[G\] defined by: \[G(x)\] is a func whose
%	value at \[y\] is \[F(x,y)\].
%	Thus, after executing the following code, \[curry(F)\] will
%	be a func whose value at \[x\] is a func
%	that can be evaluated at \[y\],  and
%	\[curry(F)(x)(y)\] is \[F(x,y)\].
%
%\DS
%curry := \= func\= (F); \\
%	 \>	   \> return \= func\= (x); \\
%	 \>	   \>        \>     \> return \= func\= (y); \\
%	 \>	   \>	     \>     \>        \>     \> return F(x,y);\\
%	 \>	   \>	     \>     \>        \> end; \\
%	 \>	   \>        \> end; \\
%	 \> end; \\
%\DE
%
%	Hence, for example, if one then executed the following code,
%
%\DS
%plus := \= func\= (a,b); \\
%	\>     \>return a+b; \\
%	\> end; \\
% \\
%inc := curry(plus)(1); \\
%\DE
%	the resulting value of \[inc(3)\] would be \(4\).
	\end{enumerate}
\end{enumerate}


    \index{func tt eff}
    \index{function eff}
    \index{proc tt eff}
    \index{procedure eff}
