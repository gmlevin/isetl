\index{Interactive Line Editor sff}
\subsubsection{Brief description}
The \[left\] and \[right\] \[arrows\] will move you within a line,
permitting insertions of characters.  \[delete\] removes the character
at the cursor, \[backspace\] deletes the character left of the cursor.
The interesting feature is that the \[up\] \[arrow\] moves you back thru
the last hundred lines entered, with \[down\] \[arrow\] moving you forward.
You can't go past the last entered line.


You need to use \[!clear\] if you want to throw away your current input
(since the last \[>\]) so that you can edit it.

\noindent Example:
\begin{verbatim}
>   a := b +
>>       c +
>>  !clear
>            =up=>      c + =up=>  a := b +
>>           =up=> a := b + =up=>       c +  =edit=> c;
\end{verbatim}
The \[!clear\] had \ISETL{} throw away the earlier input, but left it
for subsequent editting.  \[=up=>\] means typing the up arrow, followed
by the new value displayed on that line.  \[=edit=>\] means editing
the line to produce the desired result.

Below is a complete description of the new editor.


\subsubsection{Default key bindings}
     The interactive line editor is an input line 
      editor  that  provides both
     line editing and a history mechanism to edit and re-enter
     previous lines.

     \ISETL{} looks in the \[ile\] initialization file.  See
     page~\pageref{.ini} for more information.
     \index{ile.ini tt}

     Not everyone wants to have to figure out  yet  another 
      initialization  file	format	so we provide a complete set of
     default bindings for all its operations.

     The following table shows the default bindings of	keys  and
     key  sequences provided by \[ile\]. These are based on the emacs
     key bindings for similar operations.


\begin{tabular}{@{\tt}l l l}
    \bf Key&    \bf Effect&        \bf VMS differences\\
    del&	delete char under\\
    \caret A&	start of line&     undefined\\
    \caret B&	backward char\\
    \caret E&	end of line\\
    \caret F&	forward char\\
    \caret K&	erase to end of line\\
    \caret L&	retype line\\
    \caret N&	forward history\\
    \caret P&	backward history\\
    \caret U&	erase line\\
    \caret V&	quote\\
    \caret X&	delete char under\\
\\
    delete&	delete char under&  delete char before\\
    back space&	delete char before& start of line\\
    return&	add to history\\
    line feed&	add to history\\
    home&	start of line&      undefined\\
    end&	end of line&	    undefined\\
\\
    \caret C&	interrupt\\
    \caret Z&	end of file\\
    \caret D&	end of file\\
\\
    left&	backward char\\
    right&	forward char\\
    up&		backward history\\
    down&	forward history
\end{tabular}



\subsubsection{Initialization File}
     The \[ile\] initialization file is a list of table numbers,
     characters, and actions or strings. \[ile\] has 4 action tables.
     Each action table contains an action or string for each possible
     character. \[ile\] decides what to do with a character by looking it
     up in the table and executing the action associated with the
     character or by passing the string one character at a time into
     \[ile\] as if it had been typed by the user.  Normally only table~0
     is used. The escape actions cause the next character to be
     looked up in a different table. The escape actions make it
     possible to map multiple character sequences to actions.

     By default, all entries in table~0 are bound to  the  insert
     action, and all entries in the other tables are bound to the
     bell  action.  User  specified   bindings	 override   these
     defaults.	The example in Table~\ref{ile.ini} is an initialization file
     that sets up the same key and delimiter bindings as the  \[ile\]
     default bindings.


\begin{table}
\begin{quote}\begin{verbatim}
0\177=delete_char_under
0^@=escape_3
0^A=start_of_line
0^B=backward_char
0^C=pass_thru
0^D=pass_thru
0^E=end_of_line
0^F=forward_char
0^J=add_to_history
0^H=delete_char
0^K=erase_to_end_of_line
0^L=retype_line
0^M=add_to_history
0^N=forward_history
0^P=backward_history
0^U=erase_line
0^V=quote
0^X=delete_char_under
0^Z=pass_thru
0^[=escape_1

1[=escape_2

2A=backward_history
2B=forward_history
2C=forward_char
2D=backward_char

3\107=start_of_line
3\110=backward_history
3\113=backward_char
3\115=forward_char
3\117=end_of_line
3\120=forward_history
3\123=delete_char_under
\end{verbatim}\end{quote}
\caption{Example \[ile.ini\] file\label{ile.ini}}
\end{table}

     The first character on each key binding line is the index of
     the  table to place the key binding in. Valid values for the
     index are 0, 1, 2, and 3.

     The second character on the line is either the character to bind
     or an indicator that tells how to find out what character to
     bind.  If the second character is any	character besides
     `\[\caret \]'
     or `\[\esc\]' then the action is bound to that character.

     If the second character on the line is  `\[\caret \]'  then
     the  next
     character is taken as the name of a control character. So
     \[\caret H\]
     is backspace and \[\caret [\] is escape.

     If the second character on the line is a `\[\esc\]' and	the
     next character is a digit between 0 and 7 the the following
     characters are interpreted as an octal number that	indicates
     which character to bind	 the action to. If the character
     immediately after the `\[\esc\]' is not an octal	 digit	then the
     action is	 bound to that character. For example, to get the
     `\[\caret \]'
     character you would use `\[\esc\caret \]'.

     The next character on the line is always `\[=\]'. Following  the
     equal sign is the name of an action or a string. The actions
     are defined in the following table.

\subsubsection{Actions}
\begin{description}
     \item[\[bell\]]		 Send a bell (\[\caret G\]) character to the
     terminal.	 Hopefully the	 bell will ring.
			 This action is a nice way  to	tell  the
			 user  that  an	 invalid sequence of keys
			 has been typed.


     \item[\[insert\]]		  Insert  the  character  into	the  edit
			 buffer. If there are already 75 characters
     in the buffer \[ile\] will	beep and
			 refuse	 to put the character in the
			 buffer.


     \item[\[delete\_char\]]	  Delete the character	directly  to  the
			 left of the cursor from the edit buffer.


     \item[\[delete\_char\_under\]]	  Delete the character under  the  cursor
			 from the edit buffer.


     \item[\[quote\]]		  The next character  to  come	into  \[ile\]
			 will  be  inserted into the edit buffer.
			 This allows you to put	 characters  into
			 the  edit  buffer  that  are bound to an
			 action other than insert.


     \item[\[escape\_1\]]		  Look up the next  character  in  action
			 table~1 instead of action table~0.


     \item[\[escape\_2\]]		  Look up the next  character  in  action
			 table~2 instead of action table~0.


     \item[\[escape\_3\]]		  Look up the next  character  in  action
			 table~3 instead of action table~0.



     \item[\[start\_of\_line\]]	 Move the cursor to the left most character in
     the edit buffer.


     \item[\[backward\_char\]]	  Move the cursor to the left one character.


     \item[\[end\_of\_line\]]	  Move the cursor past the last character
			 in the edit buffer.


     \item[\[forward\_char\]]	  Move the cursor to the right one  character.


     \item[\[add\_to\_history\]]	  Add the contents of the edit buffer  to
			 the  history  buffer  and  pass the line
			 along to the program running under \[ile\].


     \item[\[erase\_line\]]		  Clear the line. Erase all characters on
			 the line.


     \item[\[erase\_to\_end\_of\_line\]]			 Delete the character
     under the cursor
			 and  all  character  to  the left of the
			 cursor from the edit buffer.


     \item[\[retype\_line\]]	 Retype the contents of the current edit
			 buffer.  This	is handy when system messages
     or other asynchronous output has
			 garbled the input line.

     \item[\[forward\_history\]]	  Display the next entry in  the  history
			 buffer.  If  you are already at the most
			 recent entry display a	 blank	line.  If
			 you  try  to  go  forward past the blank
			 line this command will beep at you.


     \item[\[backward\_history\]]	 Display the previous entry in the
     history buffer.	If there are	no older
			 entries in the buffer, beep.

\end{description}

\subsubsection{Strings}
     In addition to being able to bind a character sequence to an
     action \[ile\] allows characters sequences	 to be	 bound to
     strings of characters. When a string is invoked the characters in
     the string are treated as if they were typed by the user. For
     example, if the line:
\begin{center}
	  \verb|0^G=ring^Ma^Mbell^M|
\end{center}

     was in your \[ile.ini\] file, typing control G would cause  three
     lines  to	be  typed  as  if  the user typed them. Using the
     default bindings, unless there is a \[\caret J\] or \[\caret M\] in  the  string
     the string will be inserted in the current line but not sent
     along until the  user actually presses return.


\subsubsection{Error Messages}
     When \[ile\] encounters errors it  prints  a  message	and  terminates.	\[ile\]  can print several standard error message. It
     can also print a few messages that are specific to \[ile\].

\begin{itemize}
\item     \[ile: '=' missing on line \#\]

			 In a character binding line you left out
			 the `\[=\]' character. Or, you did something
			 that confused	the  initialization  file
			 reader	 into thinking there should be an
			 `\[=\]' where you didn't think there  should
			 be one.


\item     \[ile: error in initialization file on line \#\]

			 This means that the first character of a
			 character  binding line wasn't a newline
			 or a~0, 1, 2,  or  3.	It  could
			 also  mean  that the initialization file
			 reader is confused.

\end{itemize}



     A misspelled action name in an \[ile.ini\] will be  treated  as	a
     string.   This  means that typing the sequence of characters
     that should  invoke  the  action  will  actually  cause  the
     misspelled name to be inserted in the input line.

\subsubsection{Copyright}

\[ile\] and this documentation was adapted from the program called
\[ile\].  Permission to modify and distribute the program and its
documentation is granted, subject to the inclusion of its copyright
notice, which has been reproduced at the front of this manual.
\index{Interactive Line Editor eff}
