% Process with LaTeX
% For appendix, comment out lines ending %R
% For report, comment out lines ending %A


% \documentstyle[dvidoc,12pt]{article}
% \newcommand\longpage{
% \raggedbottom
% \topmargin -0.3in
% \textheight 9in
% \headheight .25in
% \headsep .3in
% \renewcommand{\dbltopfraction}{.8}
% \setcounter{dbltopnumber}{3}
% }
% \longpage
% \raggedright

\documentstyle[twoside,titlepage]{article}	%R
\newcommand{\ISETL}{{\tt ISETL\@}}
\newcommand{\SETL} {{\tt  SETL\@}}
\newcommand\twoup{
\topmargin	0.00in
\oddsidemargin	0.00in
\evensidemargin	0.00in
\headheight	0.25in
\headsep	0.20in
\textheight	6.50in
\pagestyle{headings}
}
\twoup

	\setcounter{secnumdepth}{3}	%R
	\setcounter{tocdepth}{2}
	\pagestyle{headings}	%R
	\newcommand{\XX}[1]{#1}	%R
	\newcommand{\TT}[1]{\[#1\]}	%R
	\makeindex

	\title{ An Introduction to \ISETL\\Version 3.0}	%R
	\author{Gary Marc Levin \\
	Bitnet: gary@clutx\\	%R
	Internet: gary@clutx.clarkson.edu	%R
	}	%R

	\newcommand{\DS}{\begin{center}\begin{minipage}{10in}\tt\begin{tabbing}}	%R
	\newcommand{\DE}{\end{tabbing}\end{minipage}\end{center}}	%R
	
	\newcommand{\CS}{\begin{center}}	%R
	\newcommand{\CE}{\end{center}}	%R
	
% Include the following section for appendix.
% Comment out above section.

%	\setcounter{secnumdepth}{3}	%A
%	\newcommand{\XX}[1]{#1\index{#1}}	%A
%	\newcommand{\TT}[1]{\[#1\]\key{#1}}	%A

%	\appendix	%A
%	\chapter{\ISETL{} Definition}	%A

%  Common code

\hyphenation{bool-ean}
\renewcommand{\[}{\begingroup \tt\obeyspaces}
\renewcommand{\]}{\endgroup}
\renewcommand\div{\mskip-\medmuskip\mkern5mu
  \mathbin{\rm div}\penalty900\mkern5mu\mskip-\medmuskip}
\newcommand\mod{\mskip-\medmuskip\mkern5mu
  \mathbin{\rm mod}\penalty900\mkern5mu\mskip-\medmuskip}
\newcommand{\esc}{\mbox{$\backslash$}}
\newcommand{\caret}{\mbox{\^\,}}

\newif\ifafterhead\afterheadfalse

\newsavebox{\head}
\newcommand{\key}[1]{\index{#1 tt}}
\newskip{\beforeHEAD}
\beforeHEAD=\bigskipamount
\newskip{\beforeRULE}
\beforeRULE=\medskipamount

\makeatletter
\newcommand{\HEAD}[1]{
	\if@nobreak\relax\else\vspace{\beforeHEAD}\fi
	\afterheadtrue
		  \key{#1}
		  \sbox{\head}{\hspace{0.25in}{\tt #1}\ $\rightarrow$}}
\newcommand{\RULE}[1]{\ifafterhead\nopagebreak\else\vspace{\beforeRULE}\fi
		    \afterheadfalse
		    \ifvmode\relax\else\par\fi
		  \noindent
		   \usebox{\head} {\tt #1}}
\newcommand{\COMMENT}[1]{
		    \nopagebreak
		      \begin{list}{}
			 {\@beginparpenalty\@M
			  \setlength{\topsep}{-2pt}
			  \setlength{\parsep}{2pt}
			  \setlength{\leftmargin}{2\parindent}
			 }
			 \item #1
		       \end{list}
		      }
\makeatother


\newenvironment{indented}%
{\begin{list}{}{\setlength{\leftmargin}{0.25in}\setlength{\parsep}{0in}
 \setlength{\parindent}{0in}\setlength{\topsep}{0in}\item}}%
{\end{list}\vspace{\smallskipamount}}

    \begin{document}	%R
    \maketitle	%R
    \begin{abstract}	%R
\small
    \ISETL{} is an interactive implementation of \SETL\footnote{%	%R
    \SETL{} was developed at the Courant Institute, by Schwartz.	%R
    See Schwartz, J.T., {\it et al.}	%R
    Programming with sets: An introduction to SETL\@.	%R
    Springer-Verlag, 1986.	%R
    },	%R
    a programming language built around mathematical notation	%R
    and objects,	%R
    primarily sets and functions.	%R
    It contains the usual collection of statements common to procedural	%R
    languages, but a richer set of expressions.	%R

    The objects of \ISETL{} include: integers, floating point numbers,	%R
    funcs (sub-programs),	%R
    strings, sets, and tuples (finite sequences).	%R
    The composite objects, sets and tuples, may contain any mixture of	%R
    \ISETL{} objects, nested to arbitrary depth.	%R

This introduction is intended for people who have had no
previous experience with \ISETL{},
but who are reasonably comfortable with learning
a new programming language.
Few examples are given here,
but many examples are distributed with the software.


    This documentation is a useful supplement to	%R
    {\it Learning Discrete Mathematics with \ISETL{}\/},	%R
    a discrete math text	%R
    written by Nancy \XX{Baxter}, Ed \XX{Dubinsky}, and Gary \XX{Levin},  %R
     from Springer-Verlag.	%R
    That text uses \ISETL{} as a tool for teaching discrete mathematics.	%R

\vfill

    \begin{quote}\tiny
    \begin{center}
        Copyright 1987, 1988, 1989.\\
	Gary Levin.\\
	Clarkson University.
    \end{center}

    This manual and the accompanying software may be freely copied,
    subject to the restriction that it not be sold {\em for profit\/.}
    (This would permit bulk copying and sale at cost.)
    The software is offered as-is, but we will attempt to correct
    errors in our code.

    Portions of this manual and the accompanying software are derived
    from the Interactive Line Editor, which was released with the
    following copyright restrictions.

    \begin{center}
			    COPYRIGHT 1988\\
	       Evans \& Sutherland Computer Corporation\\
			 Salt Lake City, Utah\\
			 All Rights Reserved.
\end{center}

     THE INFORMATION  IN  THIS	SOFTWARE  IS  SUBJECT  TO  CHANGE
     WITHOUT  NOTICE  AND SHOULD NOT BE CONSTRUED AS A COMMITMENT
     BY	 EVANS	\&  SUTHERLAND.	 EVANS	\&  SUTHERLAND	MAKES  NO
     REPRESENTATIONS  ABOUT  THE SUITABILITY OF THIS SOFTWARE FOR
     ANY PURPOSE.  IT IS SUPPLIED  ``AS	IS''  WITHOUT  EXPRESS  OR
     IMPLIED WARRANTY.

     IF THE SOFTWARE IS MODIFIED IN A MANNER CREATING  DERIVATIVE
     COPYRIGHT	RIGHTS,	 APPROPRIATE LEGENDS MAY BE PLACED ON THE
     DERIVATIVE WORK IN ADDITION TO THAT SET FORTH ABOVE.

     Permission	 to  use,  copy,  modify,  and	distribute   this
     software  and  its documentation for any purpose and without
     fee is hereby granted, provided  that  the	 above	copyright
     notice  appear  in	 all  copies  and that both the copyright
     notice and this permission notice appear in supporting documentation,
     and  that  the name of Evans \& Sutherland not be
     used in advertising or publicity pertaining to  distribution
     of the software without specific, written prior permission.

Written by:
     Robert C. Pendleton Evans \& Sutherland, Interactive  Systems
     Division, Salt Lake City, Utah.

     Modified for \ISETL{} by Gary Levin

     \end{quote}

    	\end{abstract}	%R
	\tableofcontents	%R
	\newpage	%R

\include{p1}

\include{p2}

\include{p3}

\include{pcompress}

\include{index}

	\end{document}	%R
