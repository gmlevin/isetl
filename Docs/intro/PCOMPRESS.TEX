\beforeHEAD=\smallskipamount
\beforeRULE=0in
\cleardoublepage

\section{The ISETL Grammar --- Compressed}
\subsection{Input at the Prompt}

\HEAD{INPUT}
\RULE{PROGRAM}
\RULE{STMT}
\RULE{EXPR ;}


\subsection{Program}

\HEAD{PROGRAM}
\RULE{program ID ; LOCALS VALUES STMTS  end ;}


\subsection{Statements}

\HEAD{STMT}
\RULE{LHS := EXPR ;}
\key{:=}
\RULE{EXPR ;}
\key{if}\key{then}\key{else}\key{elseif}\key{end}
\RULE{if EXPR then STMTS ELSE-IFS ELSE-PART end ;}
    \begin{indented}
    \HEAD{ELSE-IFS}
    \RULE{\rm ``Zero or more repetitions of \[ELSE-IF\].''}
    \HEAD{ELSE-IF}
    \RULE{elseif EXPR then STMTS}
    \HEAD{ELSE-PART}
    \RULE{else STMTS}
    \end{indented}

\RULE{for   ITERATOR do STMTS end ;}
\key{for}\key{do}\key{end}
\RULE{while EXPR do STMTS end ;}
\key{while}
\RULE{read LHS-LIST ;}
\key{read}\key{from}
\RULE{read LHS-LIST from EXPR ;}
\RULE{readf PAIR-LIST ;}
\RULE{readf PAIR-LIST to EXPR ;}
\RULE{print EXPR-LIST ;}
\key{print}\key{to}
\RULE{print EXPR-LIST to EXPR ;}
\RULE{printf PAIR-LIST ;}
\RULE{printf PAIR-LIST to EXPR ;}
\RULE{return ;}
\key{return}
\RULE{return EXPR ;}
\RULE{take LHS from LHS ;}
\key{take}\key{from} \key{fromb}\key{frome}
\RULE{take LHS frome LHS ;}
\RULE{take LHS fromb LHS ;}
\RULE{write PAIR-LIST ;}
\RULE{write PAIR-LIST to EXPR ;}
\RULE{writeln PAIR-LIST ;}
\RULE{writeln PAIR-LIST to EXPR ;}

\HEAD{STMTS}
\RULE{\rm ``One or more instances of \[STMT\].
The final semicolon is optional.''}

\HEAD{PAIR-LIST}
\RULE{\rm ``One or more instances of \[PAIR\], separated by commas.''}

\HEAD{PAIR}
\RULE{EXPR : EXPR}
\RULE{EXPR}



\subsection{Iterators}

\HEAD{ITERATOR}
\RULE{ITER-LIST}
\RULE{ITER-LIST | EXPR}

\HEAD{ITER-LIST}
\RULE{\rm ``One or more \[SIMPLE-ITERATOR\]s separated by commas.''}

\HEAD{SIMPLE-ITERATOR}
\RULE{BOUND-LIST in EXPR}
\key{in}
\RULE{BOUND = ID ( BOUND-LIST )}
\RULE{BOUND = ID \{ BOUND-LIST \} }

\HEAD{BOUND-LIST}
\RULE{\rm ``One or more instances of \[BOUND\], separated by commas.'' }

\HEAD{BOUND}
\RULE{\~{}}\key{ztilde}
\RULE{ID }
\RULE{[ BOUND-LIST ] }


\subsection{Selectors}

\HEAD{SELECTOR}
\RULE{( EXPR-LIST ) }
\RULE{\{ EXPR-LIST \} }
\RULE{( EXPR ..\ EXPR ) }
\key{..}
\RULE{( ..\ EXPR ) }
\RULE{( EXPR ..\ ) }
\RULE{( ) }


\subsection{Left Hand Sides}

\HEAD{LHS-LIST}
\RULE{\rm ``One or more instances of \[LHS\], separated by commas.''}

\HEAD{LHS}
\RULE{ID }
\RULE{@ EXPR}\key{@}
\RULE{@ ( EXPR , EXPR , ... , EXPR)}
\RULE{LHS SELECTOR }
\RULE{[ LHS-LIST ]}


\subsection{Expressions and Formers}

\HEAD{EXPR-LIST}
\RULE{\rm ``One or more instances of \[EXPR\] separated by commas.''}


\HEAD{EXPR}
\RULE{ID}\key{ID}
\RULE{INTEGER}\key{INTEGER}
\RULE{FLOATING-POINT}\key{FLOATING-POINT}
\RULE{STRING}\key{STRING}
\RULE{true}\key{true}
\RULE{false}\key{false}
\RULE{OM}\key{OM}
\RULE{newat}\key{newat}
\RULE{FUNC-CONST}\key{FUNC-CONST}
\RULE{if EXPR then EXPR ELSE-IFS ELSE-PART end ;}
\RULE{( EXPR )}
\RULE{[ FORMER ]}
\RULE{\{ FORMER \}}
    \begin{indented}
    \HEAD{FORMER}
    \RULE{\rm ``Empty''}
    \RULE{EXPR-LIST }
    \RULE{EXPR ..\ EXPR }
    \key{..}
    \RULE{EXPR , EXPR ..\ EXPR }
    \RULE{EXPR : ITERATOR }
    \end{indented}

\HEAD{EXPR}
\index{length (# tt)\\of a string}%
\index{length (# tt)\\of a tuple}%
\index{cardinality (# tt) of a set}%
\RULE{\# EXPR}
\RULE{not EXPR}\key{not}
\RULE{+ EXPR}\key{+}
\RULE{- EXPR}\key{-}
\RULE{EXPR SELECTOR}
\RULE{EXPR . ID EXPR}
\RULE{EXPR . (EXPR) EXPR}
\newpage
{\samepage
\RULE{EXPR OP EXPR}
\key{/} \key{div} \key{mod} 
\key{with} \key{less}
\key{union} \key{inter} \key{in} \key{notin} \key{subset}
\key{and} \key{or} \key{impl} \key{iff} 
\index{intersection (* tt, inter tt, zinter)}
\index{replication (* tt)\\string}
\index{replication (* tt)\\tuple}
\index{exponentiation (** tt)}
\index{concatenation (+ tt)\\string}
\index{concatenation (+ tt)\\tuple}
\index{union (+ tt, union tt, zunion)}
\index{difference (- tt) of two sets}
\index{equal}
\index{relational operators}
\nopagebreak
\COMMENT{
\nopagebreak
Possible operators (\[OP\]) are:

\begin{indented}\tt
\CS
+ - * / div mod **\\		\nopagebreak
with less\\			\nopagebreak
= /= < > <= >=\\		\nopagebreak
union inter in notin subset\\	\nopagebreak
and or impl iff			\nopagebreak
\CE
\end{indented}
}
}
\RULE{\% BINOP EXPR}\index{zpercent tt}
\index{generalize operation}
\index{compound operator (% tt)}%
\RULE{EXPR \% BINOP EXPR}
\RULE{EXPR ? EXPR}\key{?}
\RULE{choose ITER-LIST | EXPR}\key{choose}
\RULE{exists ITER-LIST | EXPR}
\key{exists}
\RULE{forall ITER-LIST | EXPR}
\key{forall}
\RULE{EXPR where DEFNS end}
\key{where}\key{end}
\RULE{EXPR @ EXPR}\key{@}

\HEAD{BINOP}
\RULE{\rm ``Any binary operator or an \[ID\] or expression in parentheses
whose value is a function of two parameters.
The \[ID\] and parenthesized expression may be preceded by a period.''}
\COMMENT{
The acceptable binary operators are:
\[+\], \[-\], \[*\], \[**\], \[union\], \[inter\],
\[/\], \[div\], \[mod\], \[with\], \[less\], \[and\], \[or\],
\[impl\].}

\HEAD{DEFNS}
\RULE{\rm ``Zero or more instances of \[DEFN\].
The final semicolon is optional.''}

\HEAD{DEFN}
\RULE{BOUND := EXPR ;}
\RULE{ID SELECTOR := EXPR ;}

\subsection{Function Constants}

\HEAD{FUNC-CONST}
\RULE{FUNC-HEAD LOCALS VALUES STMTS end}
\RULE{: ID-LIST OPT-PART -> EXPR :}

\HEAD{FUNC-HEAD}
\RULE{func ( ID-LIST OPT-PART ) ;}
\key{func}
\RULE{func ( OPT-PART ) ;}
\RULE{proc ( ID-LIST OPT-PART ) ;}
\key{proc}
\RULE{func ( OPT-PART ) ;}

\HEAD{OPT-PART}
\RULE{opt ID-LIST}
\key{opt}
\COMMENT{\rm ``May be omitted.''}

\HEAD{LOCALS}
\RULE{local ID-LIST ;}

\HEAD{VALUES}
\RULE{value ID-LIST ;}

\HEAD{ID-LIST}
\RULE{\rm ``One or more instances of \[ID\] separated by commas.''}
