\section{Running \ISETL\ on your computer}\label{extern}

It is difficult to cover all the possibilities here,
because \ISETL\ has been written so that it will run on many machines.
This means that I have no way of knowing things like memory size, number of
disks, or available software.
I will try to cover general possibilities, and leave it to you to check
with local machine experts
--- usually called {\em \GURU s} ---
for more details.
I'll put a mark in the margin like the one near the word guru to point out these
system dependent points.
Instructors may wish to check these points {\em before\/} the semester
begins.

\subsection{Creating files}\label{editor}
If you want to do more than use \ISETL\ as a calculator,
you will eventually want to create files that \ISETL\ can read.
These files may be used as input to programs you have written,
or they may be programs to be \[!include\]d.

The files should be simple text files,
also called \[ascii\] files or unformatted files.
Lines may end with CR, LF, or CR-LF, although the last case may result in
blank lines on some systems.

The problem arises of how to create such a file.
\begin{itemize}
    \item Preferably, don't use a word processor.

    Word processors put lots of extra information into files,
    so that the file can be printed in an attractive format.
    Unfortunately, each word processor has its own way of hiding that extra
    information, and there is no way that \ISETL\ can ignore it.
    More unfortunately, I have no way of knowing what needs to be done to just
    save the text.
    This is a job for your local \GURU.

    For instance, on the Mac, MacWrite has an option when you are saving
    to save text only.  If you save text only, it tries to protect you by
    asking you if you want to save the file when you exit.  If you say yes,
    it saves a formatted file.  Arrgh!

    \item On Unix and VMS, just use the standard editors --- vi, emacs, edt.
    \item On the Mac and MS-DOS, use an editor if you have one.

    Both the Mac and MSDOS now contain built-in editors.  These can be
    used to create files.  See the section on Editors in \Intro.

\end{itemize}


See \ref{system} for hints as to how to use an editor while you are using
\ISETL.


\subsection{Size of files}
There is no important limit to the length of a file,
but if you want to see your input as you \[!include\] a file,
you should take note of the width of your screen.
Each line is indented one tab.  This means that if you have 80
columns, you should limit lines to 72 columns.
The Macintosh has a smaller screen.

See \ref{echo} for an explanation of how to watch as input is read
from files.


\subsection{Standard input and output}\label{redirection}
Standard input and output (\[stdin\] and \[stdout\], respectively)
are concepts that come from Unix.
\[stdin\] is usually the keyboard, and \[stdout\] is usually the screen.
If this were all there was to it, I would just say keyboard and screen and
be done with it, but there is more.
This extra ability is useful when you want to collect a terminal session,
say to turn in as homework.
(See \ref{script} for information about how to do this in detail.)

On Unix and MS-DOS, you can re-direct \[stdin\] and/or \[stdout\] for 
{\em any\/} program.  What follows does not apply to VMS or the Mac.
See your \GURU\ for
suggestions as to how you can get the same effect.

On the command line where you type \[isetl\], 
you can also type ``\[<~afile\]''.  This means that instead of
reading from the keyboard, \[stdin\] is read from \[afile\].
Any filename may be used in place of \[afile\].
Similarly, you can type ``\[>~bfile\]'' and \[stdout\] will go to
\[bfile\] instead of the screen.
There will be {\em no visible output\/} to the screen in this case.

The characters ``\[<\]'' and ``\[>\]'' may be thought of as ``out of'' and
``in to'', respectively.

\subsection{Command line arguments --- VMS}
Command line arguments and flags may be passed in all versions of \ISETL\@,
except the Mac.
In the case of VMS, you need to do a little work first.

Define \[isetl :== \$your\$disk:[your.dir]isetl.exe\] in your
\[login.com\] and you
will be able to use command line arguments.  The leading \[\$\] makes
this a {\em foreign command\/} (a VMS term, I am told).  The rest is the
complete path to the executable version of \ISETL\@.
