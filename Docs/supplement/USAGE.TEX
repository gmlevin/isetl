\section{Using \ISETL}\label{usage}

\subsection{How to generate a terminal session}\label{script}\label{echo}
    The directive ``\[!record xyz\]'' begins appending everything that you
    type to the file \[xyz\].  This is a good idea when you are developing
    new work, or if you want to be able to show your work to someone.
    It includes everything that you typed, except for commands to edit
    past lines (see \ref{edit}).  

    When you are finished in \ISETL, you can edit \[xyz\] using your
    favorite editor (see \ref{editor}), deleting errors and dead-ends.
    In particular, remove the errors that precede \[!edit\] and the line
    with \[!edit\].  This will give you a clean script of what {\em you\/}
    did.  

    To create a terminal session where your input and \ISETL's response are
    interleaved, as you would see at the terminal, use file redirection to
    read from \[xyz\] and save it.  You need to insert \[!echo~on\] at the
    beginning of your file to tell \ISETL\ that it should print your
    input.

    \pagebreak[1]
    Example:
    \begin{description}
    \item[\[xyz\]:]
	\begin{verbatim}
!echo on
x:= 3;
y:= 4;
x+y;
x*y;
	\end{verbatim}

    \pagebreak[1]
    \item[Command:]
	\begin{verbatim}
isetl -s -d <xyz >abc
	\end{verbatim}

    \pagebreak[1]
    \item[\[abc\]:]
	\begin{verbatim}
>	x:= 3;
>	y:= 4;
>	x+y;
7;

>	x*y;
12;
	\end{verbatim}
    \end{description}

    The ``\[-s\]'' on the command line suppresses the header information
    that tells you about \ISETL\@.
    The ``\[-d\]'' tells \ISETL\ that it is receiving {\em direct\/}
    input and should not use the interactive line editor.


\subsection{Indenting rules and multiple \[ends\]}
The presentation of programs is a matter that often seems to stir
emotions usually reserved for religious questions.  The most important
point to consider is readability.  If you find that you are often
confusing where statements begin and end, you are probably are using an
inappropriate style.

I recommend that you match the beginnings and endings of statements,
and that you use the optional keywords after \[end\] when the program
gets long.  See figure~\ref{ex:indent}.

\begin{figure}[hbt]\begin{verbatim}
program nothing;
    i:= 0;
    while i < 10 do
        if i mod 3 = 5
        then print i;
        else
            print 2*i;
            print 3*i;
        end if;
        i:= i+1;
    end while;
end program;
\end{verbatim}
\caption{A simple example\label{ex:indent}}
\end{figure}

\subsection{Doing two things at once -- \ISETL\ and Editing}\label{system}
Time for the \GURU\ again.
This is very system dependent, even to the point of how much memory is
available.  

On Unix, there is usually a key to type that temporarily
stops one program, and lets you start a second (ctrl-Z is the usual choice).
\[\caret Z\] also works within \ISETL\ on MSDOS.

On VMS, the best chance is \[!system\].  This is also available
under Unix and MSDOS.
The \[!system\] directive can be used to execute a command
{\em without leaving \ISETL}\@.  ``\[!system~dir\]'' lists your
directories.   Generally, 
``\[!system~xyz\]'' is the same as executing \[xyz\] at the system prompt,
except that \ISETL\ does not lose its variables in the meanwhile.
See \Intro\ for details on how to save memory in small systems to provide
enough room to use \[!system\].

%\subsection{Programming style}
%\begin{itemize}
%    \item Avoid assignments to global variables.
%	Functions should return values, not change globals.
%\end{itemize}

\subsection{Printing output}
The simplest solution is to write to a file and then print the file
from outside of \ISETL\@.  Other possibilities exist.  See
\ref{script} and \ref{redirection}.

\subsection{Formatting output}
This is covered in \ref{lang}--\ref{write}.

\subsection{\ISETL\ for beginners}
I will consider beginners to be those who are generally
writing one-liners, simple expressions for evaluation.

\subsubsection{\[!record\]}
When you begin, it is probably best to have \[!record history\] in your
\[isetl.ini\] file.  This will keep a copy of what you type in the file
\[history\].  This is useful for showing to others when something doesn't
work.  There is little more frustrating than to say, ``This is what I did,
and it didn't work.'' only to have the other person try the same thing and
have it succeed.  If you \[!record\] everything, you can find out exactly
what you did do (probably not as you remembered it).

\subsubsection{\[!edit\]}\label{edit}
Most systems no longer have or need \[!edit\], it having been replaced
by the interactive line editor.  If your system does not have \[ILE\],
read on (the header announces the presence of \[ILE\]).
If you prefer \[!edit\], you can use it by including the \[-e\] switch.

When you are typing one-liners, and even three- and four-liners,
you will often get a character wrong and not feel like typing the
whole thing again.
Until the next \[>\] prompt appears, you have a chance to correct
your input with \[!edit\].  You will be offered the input as
it is and given the choice of accepting it or modifying it.
\begin{itemize}
\item To accept it, just type RETURN, entering a blank line as requested.
    (The message is \[!Enter string to replace (blank line to execute)\],
    which we have shortened in the example.)
\item To modify it takes two steps.
    \begin{enumerate}
    \item Enter a string to be replaced.
    \item Enter the string to be used as the replacement.
    \end{enumerate}
\end{itemize}

\begin{figure}[btp]\begin{verbatim}
>       1+2+3+;
Syntax error, line 1.
Error at or before end of:
1+2+3+;

Enter '!clear' or '!edit'
>>      !edit
1+2+3+;
!Enter string to replace (blank line...)
+;
!Enter replacement
+4;
1+2+3+4;
!Enter string to replace (blank line...)

10;
>       1+2
>>      !edit
1+2
!Enter string to replace (blank line...)
1
!Enter replacement
5
5+2
!Enter string to replace (blank line...)

>>      ;
7;
\end{verbatim}
\caption{Using \[!edit\]\label{ex:edit}}
\end{figure}

See figure~\ref{ex:edit}.
In the first case, we made an error and corrected it.
In the second case, we changed our mind and changed the 
input to match.
Try typing this to \ISETL\ to see exactly which parts are
your input and which parts are printed by \ISETL\@.

\subsubsection{\[!verbose\]}
I'd recommend turning ``\[!verbose~on\]''.  This will give more information
in the error messages, at the possible cost of printing large sets or
tuples.  As a beginner, you need all the information \ISETL\ can give you.


\subsection{\ISETL\ for intermediates}
An intermediate is someone who is now writing small programs,
too much to type each time you start again.

\subsubsection{\[!include\]}
Put your programs into files using an editor (see \ref{editor}).
You can then use ``\[!include~file.x\]'' to insert \[file.x\] as if you had
typed it at the keyboard, but without the strain on the fingers.



\subsubsection{\[!verbose\]}
I'd recommend turning ``\[!verbose~off\]''.
Most of the time, you will have enough of an idea what is wrong just from
the operation that got the error.
You can always turn it back on and repeat the error if needed.

\subsubsection{\[!watch\]}
With big programs can come big errors.
\[!watch\] can provide a trace of important changes that occur in your
program: assignments and function evaluations.

\begin{figure}[bpt] \begin{verbatim}
>       f := func(x); 
>>               return x+3;
>>           end func;
>       !watch i f
!'i' watched
!'f' watched
>       i := f(5);
! Evaluate: f(5);
! f returns: 8;
! i := 8;
\end{verbatim} 
\caption{Using \[!watch\]\label{ex:watch}}
\end{figure}
At the top level (as in figure~\ref{ex:watch}),
this is not very helpful, but inside a larger
program, this will let you see each time \[f\] is called, what its
arguments were, and what it returned.
You also see whenever \[i\] is assigned a value.

\subsubsection{Misspelled identifiers}
As programs get larger, there are more chances for things to go wrong.
A common mistake that makes you appreciate \[Pascal\]'s pickiness is
misspelling an identifier.  Any identifier is legal, so there are no syntax
errors here.  The problem is that if you misspell the name on the left of
an assignment, the variable of interest is undefined and if you misspell
the name in an expression, the misspelling is undefined.
(Let's not think about the problems when half the time you spell it one way
and half the time the other.)

The directive \[!ids\] lists identifiers that you have used that are
defined.
\[!oms\] lists those identifiers that you have used that are
undefined (have the value \[om\]).  From these lists, you can often find a
name that you didn't expect and thereby track down the problem.


\subsubsection{Debugging}
Debugging is described in detail in \Intro.

\subsection{\ISETL\ for experts}

Experts tend to find their own way of working.
Much of the decision depends on what tools are available
and how comfortable you are with your computer, separate from \ISETL\@.

I find that I tend to use \ISETL\ more in the style of a compiler,
editing files and having \ISETL\ read them in using the command line
form or \[!include\], only typing short expressions at the keyboard
for testing or evaluating functions.

Being on Unix, I use the Unix means of stopping \ISETL, rather than
\[!system\].  

It will be interesting to have others report how they make use of
\ISETL\@.

Another option that I am finding useful under Unix is to run \[emacs\]
and run \ISETL\ under the \[emacs\] shell.  This gives me a full
editor.  If you are an \[emacs\] user, just remember to run
\[isetl~-d\], so that the \[ILE\] is not active also.  The
availability of programs like \[emacs\] is the reason I resisted
introducing an editor into \ISETL; it really does not belong there.

\subsubsection{Garbage collection}
If your program seems to run very slowly, getting slower as the
program continues to run, it may be using most of its
available memory.
The directive \[!gc on\] will print information about the available
memory and how it is being collected each time \ISETL\ runs out of
space.  If it collects frequently, or the amount of space reclaimed is
small, increase the allocated memory with \[!allocate\].  If you are
at the \ISETL\ memory limit, you can increase it with \[!memory\], up
to the system limits.

On the Mac, under MultiFinder, you may need to adjust the memory usage
information in the Info window.  See a Mac \GURU.

