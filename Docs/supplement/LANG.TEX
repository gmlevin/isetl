\newcounter{question}
\newcommand{\Q}{\pagebreak[1]\vspace{0pt plus 40pt}
                \refstepcounter{question}\item[Q\arabic{question}: ]}
\newcommand{\A}{\pagebreak[0]\item[A\arabic{question}: ]}
\renewcommand{\thequestion}{(Q\arabic{question})}
\section{Questions about the language}\label{lang}

\begin{description}

\Q  Exactly when does one need to put in ``\[;\]''
    and what is it telling \ISETL?

\A  This is probably the most commonly asked question about any programming
    language that I know of, with the exceptions of \[Lisp\] and
    \[Fortran\] which don't use semicolons.

    Fortunately, the answer is easy for \ISETL;  semicolons go after every
    statement.  The semicolon tells \ISETL\ that you are finished with the
    statement.  At the prompt, where you can simply type expressions,
    the importance of the semicolon can be easily seen.  If statements and
    expressions ended at the end of the line, like \[Fortran\], the
    following would not work, instead printing \[6\] after the first line.

    \begin{quote}\begin{verbatim}
>       1+2+3
>>      +4;
10;
    \end{verbatim}\end{quote}

    Bowing to popular pressure, I have made the final semicolon in a
    list of statements optional.  This means that Pascal programmers
    will find that their intuition works fine.  All previously correct
    programs still work.  

    
\Q How is \[x+y;\] different from \[print x+y;\] ?

\A This depends on where you write them.  

\begin{itemize}
    \item
    At the prompt, an input that is just an expression implies that you
    want the value printed.  Therefore, at the top level, there is no
    difference.
    
    \item
    Inside functions or programs, \[x+y;\] simply evaluates \[x+y\], producing
    no output.  Only \[print x+y;\] would let you see the result.
\end{itemize}


\Q  If I assign an expression to \[B\], does \ISETL\ evaluate the
    expression immediately and just store the calculated value
    in \[B\], or does it wait to evaluate the expression 
    until we enter \[B;\] ?


\A \ISETL\ is a value-oriented language and as such you should think of
    expressions as being evaluated when they are used.  Therefore, the
    expression assigned to \[B\] is evaluated and the value stored.

\Q    What is the difference between \[program\] and \[func\]?

\A  A \[program\] is a collection of statements to be executed.
    It has no name, takes no arguments, executes once.  Moreover,
    there is no object in \ISETL\ that corresponds to a program ---
    that is, you can't store a program in a variable or 
    as an element of a set.

    A \[func\] is an \ISETL\ object, and as such can be used in many more
    ways.  

    The program is somewhat like the human appendix, a feature left over
    from the evolutionary predecessor, \[SETL\]\@.  It does serve a purpose
    until such time as you are ready to deal with \[func\]s.  Consider the
    difference between the following two sequences of input.

\begin{quote}\begin{verbatim}
>       program example;
>>          read x;
>>          read y;
>>          print [x,y];
>>      end program;
?       5; 6;
[5, 6];

>       
>       read x; 5;
>       read y; 6;
>       print [x,y];
[5, 6];
\end{verbatim}\end{quote}

    With a program, the input follows the program.
    With separate statements, the input must follow the \[read\].

\Q  What is pretty-printing?\label{pp}

\A  When you print using \[print\], the output is generally in the form of
    constant expressions, terminated by semicolons.  (Generally, because
    some objects don't have a constant form: atoms, files, \[func\]s.)

    The large structured objects, like sets and tuples, are printed so that
    the indentation and line breaks respect the structure of the object.
    Consider:
\begin{quote}\begin{verbatim}
>       [ [i,i**2]: i in [100..104] ];
[[100, 10000], [101, 10201],
 [102, 10404], [103, 10609],
 [104, 10816]];
\end{verbatim}\end{quote}

    Breaking the line in the middle of a pair would make it harder to read,
    so we try to format automatically.  In contrast, see \ref{write}.
    The function \[max\_line(x)\] sets the maximum number of columns of
    output in prettyprinting to \[x\]. Here I have set it to \[40\], to
    keep lines short.

\Q  What is the ``backslash convention''?\label{backslash}

\A  Not all characters in the \[ascii\] character set
    have a printed image, and some are awkward to type.
    When you type a character string, you can use the backslash convention
    to enter these special characters.
    When prettyprinting \ref{pp}, strings are printed in the same fashion.
    See \ref{write} for uses of these special characters.

    Here is a table describing these special characters:
\nopagebreak
\CS
\begin{tabular}{l l}
$\backslash$b&	backspace\\
$\backslash$f&	formfeed (new page)\\
$\backslash$n&	newline (prints as CR-LF)\\
$\backslash$q&	double quote\\
$\backslash$r&	carriage return (CR)\\
$\backslash$t&	tab\\
$\backslash${\it octal}&	character represented by {\it octal}\\
\mbox{}&			Refer to an ASCII chart for meaning.\\
$\backslash${\it other}&	{\it other} --- may be any character\\
\mbox{}&			not listed above.
\end{tabular}
\CE

In particular, \["$\backslash\backslash$"\] is a single backslash.
You may type, \["$\backslash$""\] for double quote, but the pretty
printer will print as \["$\backslash$q"\].
ASCII values are limited to \['$\backslash$001'\] to
\['$\backslash$377'\].

\begin{verbatim}
>       %+ [char(i): i in [1..127]];
"\001\002\003\004\005\006\007\b\t\n\013\f"
+"\r\016\017\020\021\022\023\024\025\026"
+"\027\030\031\032\033\034\035\036\037 !"
+"\q#$%&'()*+,-./0123456789:;<=>?@ABCDEF"
+"GHIJKLMNOPQRSTUVWXYZ[\\]^_`abcdefghijk"
+"lmnopqrstuvwxyz{|}~\177";
\end{verbatim}




\Q How do you keep from printing ``\[;\]''?
    How do you put several things on a line with appropriate spacing?
    \label{write}

\A Two questions with a common answer.  First, why the semicolon in the
    output, then how to get rid of it and produce pretty results.

    The semicolon is included in the output so that \ISETL\ output may be
    used as \ISETL\ input.  Where it makes sense, the output of \[print\]
    is a constant followed by a semicolon, suitable for \[read\].

    You can get output that is more like that of other languages by using
    \[write\].  Use \[write\] in place of \[print\] (or \[print\ldots
    to\ldots\]). It prints:
    \begin{itemize}
	\item numbers without the semicolon
	\item strings without the quotes, semicolon, 
		and the backslash convention (see \ref{backslash}).  
	\item the elements of tuples and sets,
		but without any punctuation ---
		no semicolons, no braces or brackets, no commas.
    \end{itemize}
    As in Pascal, \[writeln\] is like \[write\] with a newline written
    at the end.

    Example:\nopagebreak
\begin{verbatim}
>       print 12, "ab\ncd",
>>                 [1,2];
12;
"ab\ncd";
[1, 2];

>       write 12, "ab\ncd"; write [1,2];
        12ab
cd         1         2
>       
\end{verbatim}

    Notice that \["\esc n"\] produced a newline and that the second
    \[write\] began on the same line as the first ended.  Once you go
    to \[write\] it is your responsibility to handle lines.
    One solution is defining ``\[newline:=~"\esc n";\]'' in your \[isetl.ini\]
    file, and then you can print \[newline\] as needed.
    The other solution is to use \[writeln\].

    Each of the types has a default number of columns that it uses for
    output.  You can override the default by
    putting ``\[:d\]'' where \[d\] is an integer expression that is the
    minimum number of columns to be used.  

\begin{center}
\begin{tabular}{l r }
    Type&	Columns\\\hline
    Float&	20\\
    Integer&	10\\
    String&	0\\
    Anything else&	10
\end{tabular}
\end{center}

\begin{quote}\begin{verbatim}
>       newline := "\n";
>       write 1,2,3,newline;
\end{verbatim}\begin{verbatim*}
         1         2         3
\end{verbatim*}\begin{verbatim}
>       writeln 1:3, 2:3, 3:3;
\end{verbatim}\begin{verbatim*}
  1  2  3
\end{verbatim*}\end{quote}

    There are more possibilities for the format, to handle floating point
    numbers and structured data.  See \Intro\ for the complete story.


    
\Q  How can I read inputs without needing the semicolon?\label{readf}

\A  Just as there is \[printf\]\footnote{%
    \[printf\] is a synonym for \[write\].}, we provide \[readf\].
    You can check \Intro\ for the additional possibilities, but the two
    important ones are: 
    \begin{itemize}
    \item \[readf x;\] --- This reads the next sequence of non-white
	characters (skipping blanks, newlines, tabs, etc.).  If the
	sequence can be interpreted as a number, that is what is read.
	If it cannot be interpreted as a number, it is treated as a string.
    \item \[readf x:-1;\] --- This reads one character and assigns \[x\]
	that one character string.
    \end{itemize}
    Examples:
\begin{quote}\begin{verbatim}
>       readf x,y,z,a:-1,b:-1,c:-1;
123 xyz 123; 12
>       print [x,y,z,a,b,c];
[123, "xyz", "123;", " ", "1", "2"];
\end{verbatim}\end{quote}

    Notice:
    \begin{itemize}
    \item There is no prompt for \[readf\].  You should print your own.
    \item \[123;\] was read as a string.
    \item \[a\] was assigned \["~"\], the character {\em immediately\/}
	following ``\[123;\]''.
    \end{itemize}



\Q  How can I write a program which, when I choose a specific integer \[n\],
    would make an assignment such as \[Zn:={0..n-1}\], that is, part of the
    identifier name depends on my choice of \[n\]?  For example, when I enter
    \[MAKEZN(137);\] it assigns to \[Z137\]
    the correct set and also constructs the
    addition and multiplication mod 137 with appropriately chosen
    identifiers (\[a\] and \[m\]?).

\A  You can't do this.  You can of course have \[Z(137):= {0..137-1};\]
    and similarly for \[a(137)\] and \[m(137)\].
    Identifiers aren't \ISETL\ objects,
    so you can't perform operations on them.

    I'd recommend that \[makeZn\] return a 3-tuple and you assign the
    results as you wish.  It is not a good idea to assign to global
    variables, although it is legal.

\begin{quote}\begin{verbatim}
[Z(137),a(137),m(137)] := makeZn(137);
\end{verbatim}\end{quote}


\Q What is the purpose of having tuples start at origins other than 1?

\A A good example is polynomials, where
$$ p(x) = \sum_{i=0}^k a_i x^i $$
Here, the tuple \(a\) naturally begins at 0, so that the indices and
exponents correspond.

You can now define

\begin{tgrind}
\File{},{11:20},{Oct  2 1989}
\L{\LB{p := 0@[1,2,4];}}
\L{\LB{\C{}\$ Represents 1 + 2*x + 4*x*x}}
\CE{}\L{\LB{eval := : x,p \-\> \%+[ai*x**i : ai=p(i)] :;}}
\end{tgrind}
%	p := 0@[1,2,4];
%	$ Represents 1 + 2*x + 4*x*x
%	eval := : x,p -> %+[ai*x**i : ai=p(i)] :;


The definition of \[eval\] isn't quite right, because \[eval(p,0)\]
will compute \[0**0\].  The correct form is left as an exercise to the
reader.  Also left to the reader is the question why one sees the
summation in some texts and not others.
\end{description}
