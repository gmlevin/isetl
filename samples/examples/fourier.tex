\documentstyle[12pt]{report}
\pagestyle{empty}
\begin{document}
\section*{Partial Sums of Fourier Series}
While one can think of many programs that would be useful
in connection with teaching MA331, Fourier Series and
Boundary Value Problems, the following example might serve
to identify the basic capabilities one would want.

Specify a function $f(x)$ of a single variable $x$, defined over
a specified domain, say $-a < x < a$.  We wish first to compute
the Fourier coefficients 
\begin{eqnarray*} 
a_{0} & = & \frac{1}{2a} \int_{-a}^{a} f(x) dx \\
a_{n} & = & \frac{1}{a} \int_{-a}^{a} f(x) \cos{\frac{n\pi x}{a}} dx \\
b_{n} & = & \frac{1}{a} \int_{-a}^{a} f(x) \sin{\frac{n\pi x}{a}} dx 
\end{eqnarray*}
for $n=1,2,3,\ldots$.  Then, we wish to evaluate the sum of the
first $N$ terms of the Fourier series
$$f(x) \sim a_{0} + \sum_{n=1}^{\infty} \{
a_{n} \cos{\frac{n\pi x}{a}} + b_{n} \sin{\frac{n\pi x}{a}}
\}.$$  That is, replace $\infty$ by $N$.

The function that results from performing the partial sum of the
Fourier series should be plotted on the same axes as the
original function $f(x)$.  It will look something like the
original function, but probably will have some wiggles where the
original function was smooth.

There are lots of other things I can think of doing with Fourier
series, but if I see how one would do this problem using ISETL,
I should be able to figure out how to do other things.

Thanks,

Fred
\end{document}


